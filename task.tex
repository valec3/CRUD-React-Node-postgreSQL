\documentclass[12pt]{article}
	
\usepackage[margin=1in]{geometry}		% For setting margins
\usepackage{amsmath}				% For Math
\usepackage{fancyhdr}				% For fancy header/footer
\usepackage{graphicx}				% For including figure/image
\usepackage{cancel}				% To use the slash to cancel out stuff in work
\usepackage[spanish]{babel}
\usepackage{hyperref}
\usepackage{enumitem}
\usepackage{float}
\usepackage[utf8]{inputenc}

%%%%%%%%%%%%%%%%%%%%%%
% Set up fancy header/footer
\pagestyle{fancy}
\fancyhead[LO,L]{Base de Datos}
\fancyhead[CO,C]{SIA 301 - Views in SQL}
% fecha
\fancyhead[RO,R]{\today}
\fancyfoot[LO,L]{}
\fancyfoot[CO,C]{\thepage}
\fancyfoot[RO,R]{}
\renewcommand{\headrulewidth}{0.4pt}
\renewcommand{\footrulewidth}{0.4pt}
% secciones con 0.1
%\renewcommand{\thesection}{0.\arabic{section}}
%%%%%%%%%%%%%%%%%%%%%%

\begin{document}
% PLANTILLA INICIAL
\noindent \textbf{Universidad Nacional del Altiplano\\
Docente: } Fred Torres Cruz\\
\textbf{Estudiante:} Maye Mamani Victor Raul

\vspace{2mm}
\noindent\textbf{Trabajo Encargado N°8}\\
% %%%%%%%%%%%%%%%%%%%%%%%%%%%%%%%%%%%%%%%%%%%%%%%%%%
%%%%%%%%%%%%%%%%%%%%%%%%%%%%%%%%%%%%%%%%%%%%%%%%%%%%%

%%%%%%%%%% CONTENIDO DE LA TAREA
%%%%%%%%%%%%%%%%%%%%%%%%%%%%%%%%%%%%
%%%%%%%%%%%%%%%%%%%%%%%%%%%%%%%%%%%%%%%%%%%%%%%
%%%%%%%%%%%%%%%%%%%%%%%%%%%%%%%%%%%%%%%%%%%%%%%%%%%%%%%%%%%%%%%

\section{Login y registro de usuarios}
\subsection{Modelo de datos}
\large{Modelo de datos para el usuario}
\large{Tabla de usuarios}
\begin{verbatim}
CREATE TABLE IF NOT EXISTS users (
    usuario_id SERIAL PRIMARY KEY,
    username VARCHAR(100),
    email VARCHAR(100) UNIQUE,
    password VARCHAR(100),
    rol VARCHAR(100) DEFAULT 'user'
);
\end{verbatim}

\large{Vista para el login}
\begin{verbatim}
CREATE OR REPLACE VIEW vw_usuario_login AS
SELECT usuario_id, email, password FROM users;
\end{verbatim}


\subsection{Registro}
\large{Interfaz de registro}
\begin{figure}[H]
    \centering
    \includegraphics[width=1\textwidth]{img/registro-ui.jpg}
\end{figure}
\large{Uso de vistas para el registro}
\begin{figure}[H]
    \centering
    \includegraphics[width=1\textwidth]{img/registro-view.jpg}
\end{figure}
Se consulta la vista 'vw usuario login' para el registro de usuarios, y asi verificar si el usuario ya existe usando solo informacion de la vista y no de la tabla de usuarios.\\
Una vez verificado esto se procede a insertar el usuario en la tabla de usuarios.

\subsection{Login}
\large{Interfaz de login}
\begin{figure}[H]
    \centering
    \includegraphics[width=0.8\textwidth]{img/login-ui.jpg}
\end{figure}
\large{Uso de vistas para el login}
\begin{figure}[H]
    \centering
    \includegraphics[width=1\textwidth]{img/login-view.jpg}
\end{figure}
Se consulta la vista 'vw usuario login' para el login, para evitar que se pueda acceder a la tabla de usuarios y asi tener mayor seguridad.

\section{CRUD de Facultades y Programas}
\subsection{Modelo de datos}
\large{Modelo de datos para las facultades y programas}
\begin{verbatim}
--  tabla programa
CREATE TABLE programa (
    programa_id SERIAL PRIMARY KEY,
    facultad_id INTEGER,
    nombre VARCHAR(100),
    cod_programa VARCHAR(100),
    tipo VARCHAR(10)
);

-- tabla facultad
CREATE TABLE IF NOT EXISTS facultad (
    facultad_id SERIAL PRIMARY KEY,
    nombre VARCHAR(100),
    abreviatura VARCHAR(20),
    id_area INTEGER
);
\end{verbatim}

\subsection{Interfaz y funcionalidad}

\large{Tabla de facultades}
\begin{figure}[H]
    \centering
    \includegraphics[width=1\textwidth]{img/facs.jpg}
\end{figure}
\large{Tabla de programas}
\begin{figure}[H]
    \centering
    \includegraphics[width=1\textwidth]{img/progs.jpg}
    \label{fig:my_label}
\end{figure}

\large{Crear para los programas}
\begin{figure}[H]
    \centering
    \includegraphics[width=1\textwidth]{img/progs-form.jpg}
    \label{fig:my_label}
\end{figure}
\large{Borrar para los programas}
\begin{figure}[H]
    \centering
    \includegraphics[width=1\textwidth]{img/progs-delete.jpg}
    \label{fig:my_label}
\end{figure}
\large{Actualizar para los programas}
\begin{figure}[H]
    \centering
    \includegraphics[width=1\textwidth]{img/progs-update.jpg}
    \label{fig:my_label}
\end{figure}

\large{Funciona de igual forma para las facultades}

\section{View report}
\large {Vista para el reporte}
\begin{verbatim}
CREATE OR REPLACE VIEW vw_programas_facultades AS 
SELECT ROW_NUMBER() OVER (ORDER BY dfa.facultad_id) AS numero_registro,
    dfa.abreviatura AS Abreviatura, dpr.nombre AS Programas
FROM facultad AS dfa
    INNER JOIN programa AS dpr
    ON dfa.facultad_id = dpr.facultad_id;
\end{verbatim}

\large{Reporte}
\begin{figure}[H]
    \centering
    \includegraphics[width=1\textwidth]{img/report.jpg}
    label{fig:my_label}
\end{figure}
% poner espacio
\section{Link al repositorio:} 
\url{https://github.com/valec3/CRUD-React-Node-postgreSQL}

%%%%%%%%%%%%%%%%%%%%%%%%%%%%%%%%%%%%%%%%%%%%%%%%%%%%%%%%%%%%%%
%%%%%%%%%%%%%%%%%%%%%%%%%%%%%%%%%%%%%%%%%%%%%%%%%%
%%%%%%%%%%%%%%%%%%%%%%%%%%%%%%%%%%%%
%%%%%%%%%% FIN TAREA
\section{extra}

\large{Solo usuarios logeados pueden acceder a la Interfaz de facultades y programas(dashboard)}
\large{Tambien se añadio JWT para la autenticacion de usuarios}
\large{La base de datos esta en un servidor de Vercel}

\end{document}


% PLANTILLA PARA IMAGENES
% 
% \begin{figure}[h]
%       \centering
%       \includegraphics[width=1\textwidth]{show-facu-update.jpg}
%       \caption{Facultad actualizada con exito}
%       \label{fig:mi_imagen}
%     \end{figure}


% PLANTILLA CODIGO SQL 


% codigo sql
% \begin{verbatim}

% -- Creación de la vista
% CREATE VIEW VistaEmpleadoProyecto AS
% SELECT Empleados.Nombre AS NombreEmpleado, Proyectos.Nombre AS NombreProyecto
% FROM Empleados
% JOIN Proyectos ON Empleados.IDEmpleado = Proyectos.IDEmpleado;

% -- Uso de la vista
% SELECT * FROM VistaEmpleadoProyecto;
% \end{verbatim}

% PLANTILLA TABLA
% \begin{table}[h]
% \centering
% \begin{tabular}{|l|l|l|l|l|}
% \hline
% \textbf{ID} & \textbf{Nombre} & \textbf{Apellido} & \textbf{Edad} & \textbf{Sexo} \\ \hline
% 1           & Juan            & Perez             & 25            & M             \\ \hline
% 2           & Maria           & Lopez             & 30            & F             \\ \hline
% 3           & Carlos          & Perez             & 35            & M             \\ \hline
% \end{tabular}
% \caption{Tabla de ejemplo}
% \label{tabla:1}
% \end{table}